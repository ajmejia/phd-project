\def\abstractname{Abstract}
\begin{abstract}
%In recent decades the photometric data exceeds enormously the spectroscopic data. The Sloan Digital Sky Survey (SDSS), for instance, puts to disposition of the astronomical comunity high quality photometric data, covering a sky area considerably large and reaching...
In recent decades the photometric data exceeds enormously the spectroscopic data. The Sloan Digital Sky Survey (SDSS), for instance, puts to disposition of the astronomical comunity high quality photometric data, covering a sky area considerably large and reaching depths without precedents. This work aims to assessing the physical parameters retrieved from a photometric sample of synthetic galaxies using the DINBAS algorithm. The following steps were done to achieve the established objetive: (1) Characterization of the photometric error curves of the galaxies observed by the SDSS; (2) Construction of the library of synthetic galaxies with similar observational properties than the galaxies observed by the SDSS; (3) Comparison of the recovered mass and age from the different kinds of observations using DINBAS; (4) Establishment of an equivalence limit between the photometry and the spectroscopy in terms of signal-noise relation. It was found that the mass and the age are degenerated when is used the $u'g'r'i'z'$ photometry for assessing the galaxies physical parameters employing the DINBAS algorithm. However, either the simulated photometry and spectroscopy were quite consistent in the mass and age recuperation. In all the cases, the photometry were more accurate, with uncertainties of $64\%$ and $\sim10\%$ respectively. Besides, it was afound that synthetic galaxies with worst recovered physical parameters had colors $g-r<0.2\,\text{mag}$, bluer than the found in SDSS observations, which represents an oversampling of galaxies with young populations that is characteristic of the star formation histories used. 
\end{abstract}
