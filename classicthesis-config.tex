%%%%%%%%%%%%%%%%%%%%%%%%%%%%%%%%%%%%%%%%%
% Classicthesis Typographic Thesis
% Configuration File
%
% This file has been downloaded from:
% http://www.LaTeXTemplates.com
%
% Original author:
% André Miede (http://www.miede.de) with extensive commenting changes by:
% Vel (vel@LaTeXTemplates.com)
%
% License:
% GNU General Public License (v2)
%
% Important note:
% The main lines to change in this file are in the DOCUMENT VARIABLES
% section, the rest of the file is for advanced configuration.
%
%%%%%%%%%%%%%%%%%%%%%%%%%%%%%%%%%%%%%%%%%

%----------------------------------------------------------------------------------------
%	CHARACTER ENCODING
%----------------------------------------------------------------------------------------

\PassOptionsToPackage{utf8}{inputenc}
\usepackage{inputenc}

%----------------------------------------------------------------------------------------
%	DOCUMENT VARIABLES
%	Fill in the lines below to enter your information into the thesis template
%	Each of the commands can be cited anywhere in the thesis
%----------------------------------------------------------------------------------------

% Remove drafting to get rid of the '[ Date - classicthesis version 4.0 ]' text at the bottom of every page
\PassOptionsToPackage{pdfspacing,subfig,beramono,eulerchapternumbers}{classicthesis}
% Available options: drafting parts nochapters linedheaders eulerchapternumbers beramono eulermath
% pdfspacing minionprospacing tocaligned dottedtoc manychapters listings floatperchapter subfig

\newcommand{\myTitle}{El Contenido Estelar de las Galaxias a la Resolución de J-PAS\xspace}
\newcommand{\mySubtitle}{Proyecto de candidatura para optar al título de Ph.D.}
\newcommand{\myDegree}{Licenciado (Lic.)\xspace}
\newcommand{\myName}{Alfredo J. Mejía\xspace}
\newcommand{\myProf}{Put name here\xspace}
\newcommand{\myOtherProf}{Put name here\xspace}
\newcommand{\mySupervisor}{Gladis Magris\xspace}
\newcommand{\myFaculty}{Facultad de Ciencias\xspace}
\newcommand{\myDepartment}{Posgrado de Física Fundamental\xspace}
\newcommand{\myUni}{Universidad de Los Andes\xspace}
\newcommand{\myLocation}{Centro de Investigaciones de Astronomía (CIDA), Mérida, Venezuela\xspace}
\newcommand{\myTime}{\today\xspace}
\newcommand{\myVersion}{{\color{RedOrange} \textsc{Borrador 1.0}}\xspace}

%----------------------------------------------------------------------------------------
%	USEFUL COMMANDS
%----------------------------------------------------------------------------------------

\newcommand{\ie}{i.\,e.\xspace}
\newcommand{\Ie}{I.\,e.\xspace}
\newcommand{\eg}{e.\,g.\xspace}
\newcommand{\Eg}{E.\,g.\xspace}
\newcommand{\cf}{cf.\xspace}
\newcommand{\Cf}{Cf.\xspace}
\newcommand{\versus}{\emph{versus}\xspace}

\newcommand{\note}[1]{{\color{OrangeRed}(#1)}\xspace}
\newcommand{\mr}{\textsc{\note{referencia}}\xspace}

\newcounter{dummy} % Necessary for correct hyperlinks (to index, bib, etc.)
\providecommand{\mLyX}{L\kern-.1667em\lower.25em\hbox{Y}\kern-.125emX\@}
\newlength{\abcd} % for ab..z string length calculation

\newcommand{\mlc}{\multicolumn}
\newcommand{\mlr}{\multirow}

\newcommand{\Hip}{\ensuremath{\mathbf{X}}\xspace}
\newcommand{\hip}{\ensuremath{\mathbf{x}}\xspace}
\newcommand{\hhi}{\ensuremath{\hat{\mathbf{x}}}\xspace}
\newcommand{\dat}{\ensuremath{\{\mathcal{D}_i\}}\xspace}
\newcommand{\pos}[2]{\ensuremath{\piup_N\left(#1\,\middle|\, #2\right)}\xspace}
\newcommand{\pri}[1]{\ensuremath{\piup_0\left(#1\right)}\xspace}
\newcommand{\lik}[3][]{\ensuremath{\mathcal{L}_{#1}\left(#2\,\middle|\, #3\right)}\xspace}
\newcommand{\pro}[1]{\ensuremath{Pr\left(#1\right)}\xspace}

\newcommand{\set}[1]{\ensuremath{\left\{#1\right\}}\xspace}
\newcommand{\unit}[2][]{\ensuremath{#1\,\text{#2}}\xspace}
\newcommand{\dynbas}{\textsc{DynBaS}\xspace}
\newcommand{\tgaspex}{\textsc{tgaspex}\xspace}
\newcommand{\gaspex}{\textsc{gaspex}}
\newcommand{\starlight}{\textsc{starlight}\xspace}
\newcommand{\tform}{\ensuremath{t_\text{form}}\xspace}
\newcommand{\tburst}{\ensuremath{t_\text{brote}}\xspace}
\newcommand{\tbex}{\ensuremath{t_\text{ext}}\xspace}
\newcommand{\mcont}{\ensuremath{M_\text{cont}}\xspace}
\newcommand{\mburst}{\ensuremath{M_\text{brote}}\xspace}
\newcommand{\tcut}{\ensuremath{t_\text{trunc}}\xspace}
\newcommand{\taucut}{\ensuremath{\tau_\text{trunc}}\xspace}
\newcommand{\tauv}{\ensuremath{\tau_V}\xspace}
\newcommand{\sn}{\ensuremath{S/N}\xspace}

\newcommand{\Ni}[1]{\ensuremath{N_{#1}}\xspace}
\newcommand{\Nt}[1]{\ensuremath{N_\text{#1}}\xspace}
\newcommand{\sedi}[2][L]{\ensuremath{{#1}_\lambda^{#2}}\xspace}
\newcommand{\sedt}[2][L]{\ensuremath{{#1}_\lambda^\text{#2}}\xspace}

\newcommand{\chem}[2][H]{\ensuremath{\left[\text{#2}/\text{#1}\right]}\xspace}
\newcommand{\mw}[1]{\ensuremath{\left<#1\right>_M}\xspace}
\newcommand{\lw}[1]{\ensuremath{\left<#1\right>_L}\xspace}
\newcommand{\xsun}[1]{\ensuremath{\text{#1}_\odot}}
\newcommand{\up}[2][]{\ensuremath{\text{#1}^{#2}}}

\newcommand{\logm}{\ensuremath{\log{M_\star/\xsun{M}}}\xspace}
\newcommand{\mwla}{\ensuremath{\left<\log{t_\star/\text{año}}\right>_M}\xspace}
\newcommand{\lwla}{\ensuremath{\left<\log{t_\star/\text{año}}\right>_L}\xspace}
\newcommand{\mwlz}{\ensuremath{\left<\log{Z_\star/\xsun{Z}}\right>_M}\xspace}
\newcommand{\lwlz}{\ensuremath{\left<\log{Z_\star/\xsun{Z}}\right>_L}\xspace}
\newcommand{\extv}{\ensuremath{A_V}\xspace}

\newcommand{\res}[1]{\ensuremath{\Delta#1}\xspace}
\newcommand{\rlogm}{\ensuremath{\Delta\log{M_\star}}\xspace}
\newcommand{\rmwla}{\ensuremath{\Delta\left<\log{t_\star}\right>_M}\xspace}
\newcommand{\rlwla}{\ensuremath{\Delta\left<\log{t_\star}\right>_L}\xspace}
\newcommand{\rmwlz}{\ensuremath{\Delta\left<\log{Z_\star}\right>_M}\xspace}
\newcommand{\rlwlz}{\ensuremath{\Delta\left<\log{Z_\star}\right>_L}\xspace}
\newcommand{\rextv}{\ensuremath{\Delta\extv}\xspace}

\newcommand{\dis}[1]{\ensuremath{\delta#1}\xspace}
\newcommand{\dlogm}{\ensuremath{\delta\log{M_\star}}\xspace}
\newcommand{\dmwla}{\ensuremath{\delta\left<\log{t_\star}\right>_M}\xspace}
\newcommand{\dlwla}{\ensuremath{\delta\left<\log{t_\star}\right>_L}\xspace}
\newcommand{\dmwlz}{\ensuremath{\delta\left<\log{Z_\star}\right>_M}\xspace}
\newcommand{\dlwlz}{\ensuremath{\delta\left<\log{Z_\star}\right>_L}\xspace}
\newcommand{\dextv}{\ensuremath{\delta\extv}\xspace}

\newcommand{\dig}[1]{\ensuremath{\delta_\text{G05}#1}\xspace}
\newcommand{\glogm}{\ensuremath{\delta_\text{G05}\log{M_\star}}\xspace}
\newcommand{\gmwla}{\ensuremath{\delta_\text{G05}\left<\log{t_\star}\right>_M}\xspace}
\newcommand{\glwla}{\ensuremath{\delta_\text{G05}\left<\log{t_\star}\right>_L}\xspace}
\newcommand{\gmwlz}{\ensuremath{\delta_\text{G05}\left<\log{Z_\star}\right>_M}\xspace}
\newcommand{\glwlz}{\ensuremath{\delta_\text{G05}\left<\log{Z_\star}\right>_L}\xspace}
\newcommand{\gextv}{\ensuremath{\delta_\text{G05}\extv}\xspace}

\newcommand{\Oi}{\ensuremath{\left[\text{O\,\textsc{i}}\right]\lambda 6300}\xspace}
\newcommand{\Oii}{\ensuremath{\left[\text{O\,\textsc{ii}}\right]\lambda\lambda 3726,3729}\xspace}
\newcommand{\Oiii}{\ensuremath{\left[\text{O\,\textsc{iii}}\right]\lambda\lambda 4959,5007}\xspace}
\newcommand{\Hei}{\ensuremath{\text{He\,\textsc{i}}\lambda 5876}\xspace}
\newcommand{\Nei}{\ensuremath{\text{Ne\,\textsc{i}}}\xspace}
\newcommand{\Nii}{\ensuremath{\left[\text{N\,\textsc{ii}}\right]\lambda\lambda 6548,6583}\xspace}
\newcommand{\Sii}{\ensuremath{\left[\text{S\,\textsc{ii}}\right]\lambda\lambda 6717,6731}\xspace}
\newcommand{\Caii}{\ensuremath{\text{Ca\,\textsc{ii}}}\xspace}

\newcommand{\Dn}{\ensuremath{\text{D}4000}\xspace}
\newcommand{\Ha}{\ensuremath{\text{H}\alpha}\xspace}
\newcommand{\Hb}{\ensuremath{\text{H}\beta}\xspace}
\newcommand{\Hg}{\ensuremath{\text{H}\gamma}\xspace}
\newcommand{\Hdg}{\ensuremath{\text{H}\delta_A\!+\!\text{H}\gamma_A}\xspace}
\newcommand{\MgFe}{\ensuremath{\left[\text{Mg}_2\text{Fe}\right]}\xspace}
\newcommand{\MgbFe}{\ensuremath{\left[\text{MgFe}\right]'}\xspace}

\newenvironment{changemargin}[2]{%
\begin{list}{}{%
  \setlength{\topsep}{0pt}%
  \setlength{\leftmargin}{#1}%
  \setlength{\rightmargin}{#2}%
  \setlength{\listparindent}{\parindent}%
  \setlength{\itemindent}{\parindent}%
  \setlength{\parsep}{\parskip}%
}%
\item[]}{\end{list}}

%----------------------------------------------------------------------------------------
%	PACKAGES
%----------------------------------------------------------------------------------------

%\usepackage{lipsum} % Used for inserting dummy 'Lorem ipsum' text into the template

%------------------------------------------------

\usepackage{csquotes}
\PassOptionsToPackage{%
%backend=biber, % Instead of bibtex
backend=bibtex8,bibencoding=ascii,%
language=auto,%
%style=numeric-comp,%
style=authoryear-comp, % Author 1999, 2010
bibstyle=authoryear,dashed=false, % dashed: substitute rep. author with ---
sorting=nyt, % year, name, title
maxnames=2,
maxbibnames=10, % default: 3, et al.
backref=false,%
natbib=true, % natbib compatibility mode (\citep and \citet still work)
url=false, %
doi=false, %
eprint=false %
}{biblatex}
\usepackage{biblatex}
\defcitealias{Bruzual2003}{BC03}
\defcitealias{Chen2012}{C12}
\defcitealias{Gallazzi2005}{G05}
\defcitealias{Magris2015}{M15}

\newcommand{\bc}{\citetalias{Bruzual2003}\xspace}
\newcommand{\bcxm}{\citetalias{Bruzual2003}xm\xspace}
\newcommand{\chen}{\citetalias{Chen2012}\xspace}
\newcommand{\gal}{\citetalias{Gallazzi2005}\xspace}
\newcommand{\mgr}{\citetalias{Magris2015}\xspace}

\newcommand{\stelib}{\textsc{stelib}\xspace}
\newcommand{\miles}{\textsc{miles}\xspace}

\newcommand{\defcitepalias}[1]{\mkbibparens{\citeauthor{#1}\nameyeardelim \bibhyperref[#1]{\citeyear{#1}}\postnotedelim en adelante \citetalias{#1}}\xspace}
\newcommand{\defcitetalias}[1]{\citeauthor{#1} \mkbibparens{\bibhyperref[#1]{\citeyear{#1}}\postnotedelim en adelante \citetalias{#1}}\xspace}

\DeclareUnicodeCharacter{2010}{-}

%------------------------------------------------

%\PassOptionsToPackage{ngerman,american}{babel}  % Change this to your language(s)
% Spanish languages need extra options in order to work with this template
\PassOptionsToPackage{es-tabla,spanish,es-lcroman,english}{babel}
\usepackage{babel}

 %------------------------------------------------

\PassOptionsToPackage{fleqn}{amsmath} % Math environments and more by the AMS
 \usepackage{amsmath}

 %------------------------------------------------

\PassOptionsToPackage{T1}{fontenc} % T2A for cyrillics
\usepackage{fontenc}

%------------------------------------------------

\usepackage{textcomp} % Fix warning with missing font shapes

%------------------------------------------------

\usepackage{scrhack} % Fix warnings when using KOMA with listings package

%------------------------------------------------

\usepackage{xspace} % To get the spacing after macros right

%------------------------------------------------

\usepackage{mparhack} % To get marginpar right

%------------------------------------------------

\usepackage{marginnote} % For margin years
\newcommand{\years}[1]{\marginnote{\small #1}} % New command for including margin years

%------------------------------------------------

%\usepackage{fixltx2e} % Fixes some LaTeX stuff

%------------------------------------------------

%\PassOptionsToPackage{smaller}{acronym} % Include printonlyused in the first bracket to only show acronyms used in the text
\usepackage{acronym} % Nice macros for handling all acronyms in the thesis

%\renewcommand*{\acsfont}[1]{\textssc{#1}} % For MinionPro
\renewcommand*{\aclabelfont}[1]{\acsfont{#1}}

%------------------------------------------------

\PassOptionsToPackage{pdftex}{graphicx}
\usepackage{graphicx}

%----------------------------------------------------------------------------------------
%	FLOATS: TABLES, FIGURES AND CAPTIONS SETUP
%----------------------------------------------------------------------------------------

\usepackage{multirow}
\usepackage{tabularx} % Better tables
\setlength{\extrarowheight}{3pt} % Increase table row height
\newcommand{\tableheadline}[1]{\multicolumn{1}{c}{\spacedlowsmallcaps{#1}}}
\newcommand{\myfloatalign}{\centering} % To be used with each float for alignment
\usepackage{sidecap}
\usepackage[scriptsize,bf,normal]{caption}
\usepackage{subfig}

\setlength{\belowcaptionskip}{10pt}
%----------------------------------------------------------------------------------------
%	CODE LISTINGS SETUP
%----------------------------------------------------------------------------------------

\usepackage{listings}
\lccode`~=0
%\lstset{emph={trueIndex,root},emphstyle=\color{BlueViolet}}%\underbar} % For special keywords
\lstset{language=[LaTeX]Tex,%C++ % Specify the language(s) for listings here
morekeywords={PassOptionsToPackage,selectlanguage},
keywordstyle=\color{RoyalBlue}, % Add \bfseries for bold
basicstyle=\small\ttfamily, % Makes listings a smaller font size and a different font
%identifierstyle=\color{NavyBlue}, % Color of text inside brackets
commentstyle=\color{lightgray}\ttfamily, % Color of comments
stringstyle=\rmfamily, % Font type to use for strings
numbers=left, % Change left to none to remove line numbers
numberstyle=\scriptsize, % Font size of the line numbers
stepnumber=5, % Increment of line numbers
numbersep=8pt, % Distance of line numbers from code listing
showstringspaces=false, % Sets whether spaces in strings should appear underlined
breaklines=true, % Force the code to stay in the confines of the listing box
%frameround=ftff, % Uncomment for rounded frame
%frame=single, % Frame border - none/leftline/topline/bottomline/lines/single/shadowbox/L
belowcaptionskip=.75\baselineskip % Space after the "Listing #: Desciption" text and the listing box
}

%----------------------------------------------------------------------------------------
%	HYPERREFERENCES
%----------------------------------------------------------------------------------------

\PassOptionsToPackage{pdftex,hyperfootnotes=false,pdfpagelabels}{hyperref}
\usepackage{hyperref}  % backref linktocpage pagebackref
\pdfcompresslevel=9
\pdfadjustspacing=1

\hypersetup{
% Uncomment the line below to remove all links (to references, figures, tables, etc), useful for b/w printouts
%draft,
colorlinks=true, linktocpage=true, pdfstartpage=3, pdfstartview=FitV,
% Uncomment the line below if you want to have black links (e.g. for printing black and white)
%colorlinks=false, linktocpage=false, pdfborder={0 0 0}, pdfstartpage=3, pdfstartview=FitV,
breaklinks=true, pdfpagemode=UseNone, pageanchor=true, pdfpagemode=UseOutlines,%
plainpages=false, bookmarksnumbered, bookmarksopen=true, bookmarksopenlevel=1,%
hypertexnames=true, pdfhighlight=/O,%nesting=true,%frenchlinks,%
% urlcolor=webbrown, linkcolor=RoyalBlue, citecolor=Maroon, %pagecolor=RoyalBlue,%
urlcolor=Black, linkcolor=Black, citecolor=Black, %pagecolor=Black,%
%------------------------------------------------
% PDF file meta-information
pdftitle={\myTitle},
pdfauthor={\textcopyright\ \myName, \myUni, \myFaculty},
pdfsubject={},
pdfkeywords={},
pdfcreator={pdfLaTeX},
pdfproducer={LaTeX with hyperref and classicthesis}
%------------------------------------------------
}

%----------------------------------------------------------------------------------------
%	AUTOREFERENCES SETUP
%	Redefines how references in text are prefaced for different
%	languages (e.g. "Section 1.2" or "section 1.2")
%----------------------------------------------------------------------------------------

\makeatletter
\@ifpackageloaded{babel}
{
\addto\extrasamerican{
\renewcommand*{\figureautorefname}{Figure}
\renewcommand*{\tableautorefname}{Table}
\renewcommand*{\partautorefname}{Part}
\renewcommand*{\chapterautorefname}{Chapter}
\renewcommand*{\sectionautorefname}{Section}
\renewcommand*{\subsectionautorefname}{Section}
\renewcommand*{\subsubsectionautorefname}{Section}
}
\addto\extrasngerman{
\renewcommand*{\paragraphautorefname}{Absatz}
\renewcommand*{\subparagraphautorefname}{Unterabsatz}
\renewcommand*{\footnoteautorefname}{Fu\"snote}
\renewcommand*{\FancyVerbLineautorefname}{Zeile}
\renewcommand*{\theoremautorefname}{Theorem}
\renewcommand*{\appendixautorefname}{Anhang}
\renewcommand*{\equationautorefname}{Gleichung}
\renewcommand*{\itemautorefname}{Punkt}
}
\addto\extrasspanish{
\renewcommand*{\lstlistlistingname}{Índice de pseudo-códigos}
}
\providecommand{\subfigureautorefname}{\figureautorefname} % Fix to getting autorefs for subfigures right
}{\relax}
\makeatother

%----------------------------------------------------------------------------------------

\usepackage{classicthesis}

%----------------------------------------------------------------------------------------
%	CHANGING TEXT AREA
%----------------------------------------------------------------------------------------

\linespread{1.05} % a bit more for Palatino
\areaset[current]{370pt}{760pt} % 686 (factor 2.2) + 33 head + 42 head \the\footskip
%\setlength{\marginparwidth}{7em}%
%\setlength{\marginparsep}{2em}%

%----------------------------------------------------------------------------------------
%	USING DIFFERENT FONTS
%----------------------------------------------------------------------------------------

%\usepackage[oldstylenums,notextcomp,lighttext]{kpfonts} % oldstyle notextcomp
\usepackage[osf,tt=false]{libertine}
%\usepackage[light,condensed,math]{iwona}
%\renewcommand{\sfdefault}{iwona}
%\usepackage{lmodern} % <-- no osf support :-(
%\usepackage{cfr-lm} %
%\usepackage[urw-garamond]{mathdesign} %<-- no osf support :-(
%\usepackage[default,osfigures]{opensans} % scale=0.95
%\usepackage[sfdefault]{FiraSans}
