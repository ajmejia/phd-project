\chapter{Introducción}

\label{ch:introduction}

%----------------------------------------------------------------------------------------

\section{El contexto}

En el camino hacia la construcción de una teoría que explique la formación de las estructuras
visibles en el Universo, las galaxias, que son los bloques fundamentales que conforman dicha
estructura, se han convertido en el tópico de investigación más importante de la cosmología
observacional. Aunque no existe una definición formal de lo que es una galaxia
\citep[\eg][]{Forbes2011}, esta es generalmente descrita como un sistema gravitacionalmente ligado
que resulta de la co-evolución dinámica de un halo de materia oscura y un conjunto de subestructuras
de materia bariónica. Dicha materia bariónica coexiste en múltiples fases: gas ionizado, atómico y
molecular, y en plasmas autogravitantes llamados estrellas. Aunque en esta definición clasifican la
gran \emph{mayoría} de los objetos extragalácticos observados en el presente, la falta de una
definición formal es mero reflejo de que una teoría que explique la existencia de \emph{todos} estos
objetos está ausente. El camino tiene dos sentidos. Uno consiste en una construcción \emph{ab
initio} para la formación y la evolución de las estructuras en el Universo a partir de las leyes
fundamentales de la física y de un contexto cosmológico que dicte las escalas espaciales y
temporales en las cuales los fenómenos físicos tienen lugar.

El otro sentido, el cual será el recorrido en el presente estudio, consiste en interpretar los
observables en las propiedades físicas de las galaxias observadas a distintas épocas cosmológicas.
Para ello los sondeos proveen la información necesaria en la forma de mediciones de la radiación
electromagnética que las galaxias emiten. En términos de las distintas componentes del contenido
bariónico en una galaxia, la luminosidad \emph{emitida}, en la longitud de onda
$\lambda=\lambda_\text{emi}$, en el instante $t_\text{emi}$, puede representarse simplemente como la
suma:
%
\begin{equation}\label{ec:luminosidad-integrada}
\sedt{emi}(Z;t_\text{emi}) = L_{\lambda,\star}(Z_\star;t_\text{emi}) +
                             L_{\lambda,\text{MIE}^+}(Z_{\text{MIE}^+};t_\text{emi}) -
                             L_{\lambda,\text{MIE}^-}(Z_{\text{MIE}^-};t_\text{emi}),
\end{equation}
%
donde ${L_\lambda}_\star$ es luminosidad integrada de cada estrella, y $L_{\lambda,\text{MIE}^-}$ y
$L_{\lambda,\text{MIE}^+}$ son la luminosidad absorbida/dispersada por parte del gas y del polvo
fríos, y la re-emisión por parte del gas ionizado y del polvo caliente, respectivamente, en el Medio
Interestelar (MIE). La luminosidad por supuesto depende de las propiedades físicas del medio en el
que se origina y a través del cual viajan los fotones. De manera que ésta se ha descrito en la
Ec.~\eqref{ec:luminosidad-integrada} como función del contenido químico, usualmente parametrizado
por la metalicidad (\ie, la fracción de especies químicas más pesadas que el helio) en los
interiores estelares, $Z_\star$, y de la metalicidad del MIE, $Z_\text{MIE}$; ambas a su vez
funciones del tiempo.

La cantidad que se mide en los observatorios no es la luminosidad en sí, sino el flujo, \ie la
cantidad de energía radiada por unidad de tiempo por unidad de área, $\sedt[F]{obs}$, observado en
un rango de longitudes de onda, $\lambda=\lambda_\text{obs}$. Este flujo es inevitablemente
perturbado con incertidumbres inherentes al proceso de medición debidas a las limitaciones
tecnológicas del instrumento, la técnica de medición y a las condiciones en que se realizó la misma.
A estas fuentes de incertidumbre se les denominará $\epsilon_\text{med}$. Si la radiación
electromagnética emitida es isotrópica, entonces la luminosidad en la
Ec.~\eqref{ec:luminosidad-integrada} puede reescribirse en términos del flujo medido como:
%
\begin{equation}\label{ec:flujo-integrado}
F_{\lambda_\text{obs}}^\text{obs}(Z;t_\text{emi}) = \frac{1}{1+z}\frac{L_{\lambda_\text{emi}}^\text{emi}(Z;t_\text{emi})}{4\pi D_L(z)^2} + \delta{F_{\lambda_\text{obs}}^\text{obs}}(\epsilon_\text{med}),
\end{equation}
%
donde $D_L(z)$ es la distancia que recorre un fotón desde que escapa al Medio Intergaláctico (MIG)
hasta que es observado por el detector (distancia luminosa) y
$z\equiv\lambda_\text{obs}/\lambda_\text{emi}-1$ es el corrimiento al rojo cosmológico, \ie, el
factor por el cual éste se ha enlongado hacia el rojo en longitud de onda debido a la expansión del
Universo, desde el instante de su emisión \citep{Hogg1999, Hogg2002}. Es el objetivo principal de la
astronomía extragaláctica interpretar el conjunto de observaciones de galaxias individuales,
$\set{\sedt[F]{obs}}$, provisto por los sondeos, en la Historia de Enriquecimiento Químico (HEQ),
$\zeta(Z_0,t)$, y en la Historia de Formación Estelar (HFE), $\psi(Z;t)$, con el fin de avanzar en
la construcción teórica que describa la formación y la evolución de las galaxias. Tal objetivo no es
trivial, pues no solo las incertidumbres introducidas por $\epsilon_\text{med}$ intervienen, sino
que los procesos físicos involucrados, los cuales ocurren en una gran variedad de escalas espaciales
y temporales, suponen que la astronomía extragaláctica debe fundamentarse en los pilares de la
física fundamental \citep[véase][para revisiones detalladas]{Somerville2015, Naab2017}.
Consecuentemente, las incertidumbres inherentes a cada teoría y a cada supuesto, encapsuladas en
$\epsilon_\text{teo}$, se propagarán durante el proceso de interpretación de $\set{\sedt[F]{obs}}$
en las funciones $\zeta(Z_0,t)$ y $\psi(Z;t)$.

Además de las fuentes de incertidumbres mencionadas en los párrafos anteriores, el problema de
interpretar los observables en parámetros los físicos está condicionado por dos razones
principalmente. En primer lugar está el hecho de que las observaciones solo proveen la función
$F_{\lambda_\text{obs}}(Z;t)$ evaluada en $t_\text{emi}$ en la escala temporal de cada galaxia, por
lo tanto la representación de las observaciones en el espacio de las funciones $\zeta(Z_0,t)$ y
$\psi(Z;t)$ está probablemente degenerada, \ie existe un conjunto de funciones capaces de reproducir
la misma observación. En segundo lugar está el hecho de que la evolución química ocurre de manera
diferencial en el tiempo, lo que significa que no todas las especies más pesadas que el helio están
presentes en las mismas proporciones \citep[\eg,][]{Yates2013}, de manera que $Z$ y $t_\text{emi}$
no solo son difíciles de medir, ambas están relacionadas de una forma \emph{a priori} desconocida.
Así, los métodos o técnicas utilizados para llevar a cabo la interpretación de las observaciones
$\set{\sedt[F]{obs}}$ en las funciones $\zeta(t)$ y $\psi(Z;t)$, necesariamente introducirán una
serie de suposiciones y simplificaciones, las cuales se traducirán en una fuente de incertidumbre
adicional. A esta se le denotará por $\epsilon_\text{met}$. Entonces, se puede describir cualquier
variable física, $X_\text{recuperado}$ relacionada con las funciones $\zeta(Z_0,t)$ y $\psi(Z;t)$ y
que resulte de dicha interpretación, como:
%
\begin{equation}\label{ec:uncertainties}
X_\text{recuperado} = X_\text{real} + \delta{X}(\epsilon_\text{teo},\epsilon_\text{obs},\epsilon_\text{met}),
\end{equation}
%
donde $X_\text{real}$ es el valor intrínseco real de la propiedad que se quiere determinar.

\section{El problema}

En los párrafos anteriores se ha descrito el problema principal de la astronomía extragaláctica, en
el cual está enmarcado el presente estudio, en términos de las incertidumbres que se originan en las
teorías relevantes (\eg, la teoría de formación de estructuras, la teoría del transporte radiativo,
las teorías de formación y de evolución estelar, etc.), las incertidumbres inherentes al proceso de
medición y las que tienen origen en el método de interpretación. En esta sección se delimitará la
contribución que hará el presente estudio para resolver dicho problema.

\subsection{Sobre los sondeos de galaxias}

El producto de los sondeos que será relevante en este estudio es el conjunto de medidas
$\set{\sedt[F]{obs}}$ de galaxias individuales. Se limitará el presente estudio a sondeos de
galaxias en el rango espectral óptico, los cuales además proporcionan una plétora de información
disponible públicamente \citep[\eg,][]{Wolf2003, Moles2008, Abazajian2009}. Esto se traducirá en una
minimización el impacto de las incertidumbres introducidas por los ingredientes teóricos
($\epsilon_\text{teo}$), cuyas predicciones son inciertas más allá del rango óptico del espectro.

Aún con la restricción anterior, existen muchas formas en que $X_\text{recuperado}$ puede ser
perturbado a través de $\set{\sedt[F]{obs}}$, sin embargo, este estudio se enfocará en la
contribución de la resolución espectral a dichas perturbaciones, debido a que es un parámetro
altamente variable de un sondeo a otro ($\Delta\lambda\sim1\,$---$\,\unit[1000]{\AA}$) y a que éste
puede limitar críticamente la recuperabilidad de propiedades físicas fundamentales como la edad
estelar y la metalicidad \citep[\eg,][]{MacArthur2010, Pforr2012, Mitchell2013}. En ese sentido, se
han escogido tres resoluciones representativas de los sondeos de galaxias en el rango óptico:
espectroscopía de alta resolución ($\Delta\lambda\sim\unit[1]{\AA}$) y, fotometría de banda angosta
($\Delta\lambda\sim\unit[100]{\AA}$) y de banda ancha ($\Delta\lambda\sim\unit[1000]{\AA}$). En el
siguiente capítulo se darán los detalles de los sondeos escogidos y de las características de
$\set{\sedt[F]{obs}}$.

\subsection{Sobre la síntesis espectral}

El método de interpretación más implementado en la literatura es el de síntesis espectral
\citep[pero véase \eg,][]{Chen2012}, pues ha sido diseñado especialmente para explotar el poder
estadístico provisto por los sondeos de galaxias, para muestrear el espacio de funciones
$\zeta(Z_0,t)$ y $\psi(Z;t)$ \citep[\eg,][]{Heavens2000, Kauffmann2003}. En la síntesis espectral el
conjunto de medidas para una galaxia observada, $\set{\sedt[F]{obs}}$, es comparado con la
predicción de los ingredientes físicos combinados en un modelo plausible. Así, las distintas fuentes
de incertidumbres descritas en los párrafos anteriores se convolucionan durante el proceso de
interpretación de una forma que no es trivial predecir, pues no todas las incertidumbres son
conocidas de antemano. En particular, durante este proceso, el modelo asumido por el método de
síntesis espectral para describir las funciones $\zeta(Z_0,t)$ y $\psi(Z;t)$ propaga las
incertidumbres encapsuladas en $\epsilon_\text{met}$. En este trabajo se implementará el método de
síntesis espectral no paramétrico, que tradicionalmente ha requerido conjuntos $\set{\sedt[F]{obs}}$
con un número de elementos de resolución espectral del orden de $N_\text{obs}\sim1000$ (véase el
siguiente capítulo para más detalles), para demostrar su aplicabilidad a sondeos fotométricos de
galaxias, que usualmente proporcionan conjuntos $N_\text{obs}\sim10$. Aunque el modelo propuesto
captura los fenómenos físicos no lineales representados en la Ec.~\eqref{ec:flujo-integrado} por las
contribuciones del MIE (segundo y tercer término del lado derecho de la igualdad), el enfoque
principal en este estudio será el contenido estelar de las galaxias, de manera que las escalas
temporales relevantes serán la escala de formación estelar y la de evolución estelar.

\section{El objetivo}

El principal objeto de este estudio es demostrar que el contenido estelar codificado en la DEE de
una galaxia no resuelta, puede extraerse a partir de observaciones fotométricas utilizando el método
de la síntesis espectral no paramétrico. Para esto se desarrollará un análisis sobre la propagación
de las incertidumbres durante la interpretación de $\set{\sedt[F]{obs}}$ en las funciones
$\zeta(Z_0,t)$ y $\psi(Z;t)$, haciendo énfasis en el comportamiento estadístico de las fuentes de
incertidumbre combinadas $\epsilon_\text{obs} \otimes \epsilon_\text{teo} \otimes
\epsilon_\text{met}$, como función de la resolución espectral.

\subsection{Los objetivos específicos}

\paragraph{Definir las muestras.} Se seleccionarán dos muestras de galaxias. La primera será una
muestra de galaxias sintéticas con ``medidas'' en tres resoluciones espectrales representativas de
espectroscopía de alta resolución, fotometría de banda ancha y fotometría de banda angosta, todas en
el rango óptico del  espectro. La segunda muestra será una observada usando la técnica de la
espectroscopía de alta resolución.

\paragraph{Realizar pruebas de consistencia \emph{interna}.} La muestra sintética a las distintas
resoluciones será utilizada para evaluar las incertidumbres propagadas durante la interpretación de
las observaciones en la forma de un diagnóstico \emph{interno}, donde $X_\text{real}$ es conocido de
antemano.

\paragraph{Realizar pruebas de consistencia \emph{externa}.} La fotometría tradicional (de banda
media/ancha) proporciona poca información sobre la metalicidad debido a que los rasgos espectrales
trazadores de esta propiedad son generalmente de unas decenas de \AA. Por esta razón se ha planteado
en este trabajo evaluar la recuperabilidad de la metalicidad estelar a partir de observaciones
fotométricas de banda angosta, en el marco del \emph{Javalambre Physics of the Accelerating Universe
Astrophysical Survey} (J-PAS). Para esto, la muestra real se utilizará para realizar un diagnóstico
\emph{externo} comparando los resultados del presente estudio con los presentados por otros autores.

El método para realizar la interpretación, los detalles de las muestras de datos simulados y
observados, y los procedimientos estadísticos para el análisis de los resultados se describen a
continuación.
